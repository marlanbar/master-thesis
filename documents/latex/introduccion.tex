\section{La Bioinformática y la Medicina Personalizada}

La biología y en particular la genética ha tenido un lugar destacado en el siglo XX gracias a enormes avances como el descubrimiento de la estructura molecular del ADN con el aporte de Franklin, Crick y Watson. A partir de esos hitos científicos se sucedieron grandes esfuerzos a nivel internacional como el Proyecto Genoma Humano o ENCODE, que tienen como objetivo general el mapeo del código genético humano \textbf{[CITAR]} y sus elementos funcionales \textbf{[CITAR]}. Estos trabajos, entre otros, han generado enormes volúmenes de datos que necesitan ser procesados para obtener información. La bioinformática puede ser definida entonces como la aplicación de la informática para el entendimiento y el uso efectivo de los datos biológicos. Estas tareas incluyen:

\begin{itemize}
    \setlength\itemsep{0.5em}
    \item Inferir la forma de una proteína y su función a partir de una secuencia de aminoácidos,
    \item Encontrar todos los genes y proteínas de un genoma dado,
    \item Determinar sitios en la estructura de la proteína donde moléculas de una droga determinada pueda adherirse.
\end{itemize}


\section{Conceptos Biológicos}

\subsection{El Genoma Humano}


Para definir el genoma humano primero es necesario definir que es un gen. Esta definición ha ido evolucionado con el tiempo debido a los frecuentes descubrimientos en este área \textbf{[CITAR]}.   
El gen se define como una parte del material genético que codifica una proteína. Este material genético está conformado por \textbf{nucleótidos}, tríos conformados por un azúcar, un grupo fosfato y una base nitrogenada, que puede ser Adenina (A), Timina (T), Guanina (G) o Citosina (C). 

El genoma humano está compuesto por alrededor de 20.000 genes \textbf{[CITAR]}. Cada uno de estos genes se encuentran en alguno de los 23 pares de cromosomas (moléculas de ADN) que componen el código genético de un individuo. 



En el caso del ARN la Citosina es reemplazada por el Uracilo (U).

Las partes de un gen se dividen en dos: los exones, que son la parte codificante de los genes, y los intrones, que se eliminan en el procesamiento del ARN. 
A su vez, los exones están conformado por codones, conjuntos de tres bases que configuran un único aminoácido. 


\subsection{Proteínas}

Las proteínas son parte fundamental de todo organismo vivo. Son biomoléculas formadas por cadenas de aminoácidos, y de acuerdo a su composición y forma, cumplen diferentes funciones vitales.

\subsection{Polimorfismos de un Sólo Nucleótido (SNPs)}

Durante el proceso de replicación o división de las células, pueden surgir distintos "errores", denominados mutaciones. Estas mutaciones son de vital importancia en el proceso de selección natural, pero también pueden ser la causas de distintas enfermedades. Existen diferentes tipos de mutaciones en el ADN, una de ellas siendo las SNPs, que consisten en el reemplazo de una base nucleótida por otra.


\section{Aprendizaje Automático}

El aprendizaje automático es un método computacional que consiste en aprender a partir de los datos. Este aprendizaje puede ser supervisado, en donde se utilizan ejemplos previos que están "rotulados", no supervisado, donde el objetivo es encontrar distintos grupos de ejemplos dentro de los datos. En esta tesis trabajaremos con algoritmos supervisados. A partir de ellos podemos realizar tareas diversas, que podemos agrupar en dos grandes grupos: la clasificación de nuevos datos, y la regresión, en donde el objetivo es la predicción de una variable continua. 

\subsection{Aprendizaje Supervisado}

% \subsection{Regresión Logística}
% \subsection{Random Forest}

\section{Bases de Datos Genéticas}

\section{Trabajo previo}

\section{Objetivo de este trabajo}

La predicción de enfermedades causadas por los polimorfismos de un solo nucleótido (SNPs, por sus siglas en inglés) tuvo un desarrollo constante y acelerado en los últimos años en parte gracias a nuevas técnicas de secuenciación del genoma, que permite el análisis del material genético de pacientes con mayor facilidad. Nuestro objetivo es generar un método de predicción propio a través del análisis de distintas herramientas ya existentes, reutilizando sus features más importantes, como así también agregando nuevos y explorando distintos métodos de Machine Learning que superen las métricas alcanzadas por los trabajos previos en el área.


