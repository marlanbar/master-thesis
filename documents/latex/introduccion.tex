\section{Objetivo Inicial}

La predicción de enfermedades causadas por los polimorfismos de un solo nucleótido (SNPs, por sus siglas en inglés) tuvo un desarrollo constante y acelerado en los últimos años en parte gracias a nuevas técnicas de secuenciación del genoma, que permite el análisis del material genético de pacientes con mayor facilidad. Nuestro objetivo es generar un método de predicción propio a través del análisis de distintas herramientas ya existentes, reutilizando sus features más importantes, como así también agregando nuevos y explorando distintos métodos de Machine Learning que superen las métricas alcanzadas por los trabajos previos en el área.


\section{Introducción Biológica}

\subsection{El Genoma Humano}

El genoma humano está compuesto por alrededor de 20.000 genes. Cada uno de estos genes se encuentran en alguno de los 23 pares de cromosomas (moléculas de ADN) que componen el código genético de un individuo. 
El gen se define como una parte del material genético que codifica una proteína. El origen de de muchos de ellos provienen de nuestros ancestros e incluso antiguas especies. A su vez, un gen está conformado por codones, conjuntos de tres bases que configuran un único aminoácido.


\subsection{Proteínas}

Las proteínas son parte fundamental de todo organismo vivo. Son biomoléculas formadas por cadenas de aminoácidos, y de acuerdo a su composición y forma, cumplen diferentes funciones vitales.



\section{Técnicas de Machine Learning}

\section{Trabajo previo}