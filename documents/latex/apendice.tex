\section{Estructura del proyecto}

A continuación detallamos los distintos módulos del proyecto para facilitar la replicación de los resultados obtenidos. Este se encuentra alojado en GitHub en el sitio \url{https://github.com/marlanbar/master-thesis}. 

El proyecto tiene la siguiente estructura:

\vspace{0.2cm}
\dirtree{%
.1 master-thesis.
.2 data.
.3 external.
.3 interim.
.3 processed.
.2 notebooks.
.2 results.
.3 varq.
.3 physico\_chemical.
.3 genomic.
.3 integral.
.3 varq\_integral.
.2 src.
.3 snvbox\_queries.
}
\vspace{0.2cm}

En \texttt{src} se encuentra el código necesario para obtener las variables de la tabla SNVbox y generar las variables a partir de la data externa (por ejemplo, Humsavar) que se encuentra en la carpeta \texttt{data/external}. Las variables procesadas se encuentran en \texttt{data/interim}. Luego, para cada modelo existe una notebook para generar los datasets, ubicados en \texttt{data/processed}, y otro para evaluar los algoritmos, que almacenan los scores de las variantes en cada una de sus respectivas carpetas en \texttt{results}. El trabajo fue realizado con una notebook Dell XPS 9350 (Intel Core i5 Skylake, 8GB de memoria RAM, disco SSD 128GB). 


\section{Diccionario de hiperparámetros usados}

Para los modelos usados, se usaron los siguientes diccionarios de parámetros:
\begin{itemize}
    \item Random Forest
        \begin{itemize}
            \item Profundidad del árbol (\texttt{max\_depth}): [3, 5, 7]
            \item Estimadores (\texttt{n\_estimators}): [10, 50, 100]
            \item Cantidad de máxima de variables por árbol (\texttt{max\_features}): [4, $\sqrt{n}$, 0.2*$n$] con $n$ la cantidad total de variables
        \end{itemize} 
    \item Regresión logística:
        \begin{itemize}
            \item Parámetro de regularización inverso (C): [.001, .01, .1, 1, 10, 100, 1000]
            \item Peso de las clases: [balanceado, igual a 1]
        \end{itemize}
    \item SVC:
        \begin{itemize}
            \item Parámetro de penalidad (C): [0.001, 0.10, 0.1, 10, 25, 50, 100, 1000]
            \item Gamma: [0.001, 0.10, 0.1, 10, 25, 50, 100, 1000]
        \end{itemize}
    \item XGBoost:
        \begin{itemize}
            \item Peso mínimo de las hojas: [1, 5, 10]
            \item Gamma: [0.5, 1, 1.5, 2, 5]
            \item Subsample: [0.6, 0.8, 1.0]
            \item colsample\_bytree: [0.6, 0.8, 1.0]
            \item Profundidad máxima: [3, 4, 5]
        \end{itemize}
\end{itemize}


\section{Lista de Variables de SNVBox} \label{snvbox_list}

La siguiente sección es una lista detallada de variables de SNVBox que fueron usadas en el trabajo. La lista completa junto con el esquema de la base de datos se encuentra en la wiki del proyecto SNVBox \cite{snvbox}. 

\subsubsection{Variables sobre cambios en la sustitución del Aminoácido (Estructura Primaria)}
\begin{itemize}
    \item Score BLOSUM (AABLOSUM): Score de la matriz BLOSUM 62.
    \item Carga (AACharge): El cambio en la carga resultante de cambiar el aminoácido de referencia con la mutación.
    \item Volumen (AAVolume): El cambio en el residuo resultante del cambio de aminoácido (expresado en Angstroms cúbicos).
    \item Hidrofobia (AAHydrophobicity): El cambio en hidrofobicidad resultante de la mutación.
    \item Score Grantham (AAGrantham): La distancia Grantham de la referencia a la mutación.
    \item Polaridad (Polarity): Cambio de polaridad entre la referencia y la mutación.
    \item Score Ex (AAEx): Score de la matriz EX.
    \item Score PAM250 (AAPAM250): Score de la matriz PAM250.
    \item Score MJ (AAMJ): Score de la matriz Miyagawa-Jerningan.
    \item Score VB (AAVB): Score de la matriz VB (Venkatarajan \& Braun).
    \item Transición (Transition): Frecuencia de la transición entre dos aminoácidos vecinos basado en todas las proteínas de SwissProt/TrEMBL.
\end{itemize}

\subsubsection{Variables a nivel de Proteína (sin considerar sustitución)}

\begin{itemize}
    \item BINDING: Sitio de unión.
    \item ACTIVE\_SITE: Actividad enzimática. 
    \item SITE: Sitio de aminoácido ``interesante'' en la secuencia proteica.
    \item LIPID: Unión con un lípido.
    \item METAL: Unión con un metal.
    \item CARBOHYD: Unión con un carbohidrato.
    \item DNA\_BIND: Unión con ADN.
    \item NP\_BIND: Unión con un nucleótido fosfato.
    \item CA\_BIND: Unión con calcio.
    \item DISULFID: Sitio de unión con un disulfuro.
    \item SE\_CYS: Selenocisteína.
    \item MOD\_RES: Residuo modificado.
    \item PROPEP: Propéptido.
    \item SIGNAL: Sitio de señal de localización.
    \item TRANSMEM: Proteína transmembranal.
    \item COMPBIAS: Sesgo de composición.
    \item REP: Región de repetición.
    \item MOTIF: Sitio de un motif funcional conocido.
    \item ZN\_FING: Dedo de zinc.
    \item REGIONS: Región de interés en la secuencia proteica.
    \item PPI: Interacción proteína-proteína.
    \item RNABD: Unión RNA.
    \item TF: Factor de transcripción. 
    \item LOC: Sitio que determina correcta localización celular de una proteína. 
    \item MMBRBD: Sitio que se une a la membrana celular.
\end{itemize}

% \vspace{2mm}
% \todo{
% TODO:
% Diagrama de SNVBOX
% }
% \vspace{2mm}