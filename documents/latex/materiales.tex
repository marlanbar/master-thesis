En esta sección detallaremos cada unos de los datasets usados durante nuestro trabajo, las distintas fuentes de atributos, y el esquema de clasificación usado.

\section{Análisis de los Datasets}

\subsection{Dataset VarQ}

El primer paso que tomamos fue realizar un análisis del dataset VarQ. Este dataset fue un acercamiento inicial al problema de predecir si a partir de una mutación en el gen. Para esto verificamos la calidad de cada una de las variables incluyendo el tipo. El dataset VarQ tiene 17.8k variantes, de los cuales:

\begin{itemize}
    \item 11.7 mil están catalogados como benignos.
    \item 6.1 mil están catalogados como patogénicos.
\end{itemize}

Posee 12 columnas o variables: 

\begin{itemize}
    \item MUTANT: Código identificatorio de la variante, compuesto por el código Uniprot, la posición del aminoácido donde ocurre la variación, y el cambio de aminoácido.
    \item SASA (Solvent-Accessible Surface Area): El área del aminoácido accesible por un solvente.
    \item SASA\_PERCENTAGE: El porcentaje del SASA sobre la superficie del aminoácido.
    \item BFACTOR: 
    
\end{itemize}

\subsection{Dataset Humsavar}


\section{Variables}

\subsection{ProtParam}

\subsection{VEST}

\subsection{SNVBox}


\section{Esquema de Clasificación}